% Options for packages loaded elsewhere
% Options for packages loaded elsewhere
\PassOptionsToPackage{unicode}{hyperref}
\PassOptionsToPackage{hyphens}{url}
\PassOptionsToPackage{dvipsnames,svgnames,x11names}{xcolor}
%
\documentclass[
  12pt]{article}
\usepackage{xcolor}
\usepackage{amsmath,amssymb}
\setcounter{secnumdepth}{5}
\usepackage{iftex}
\ifPDFTeX
  \usepackage[T1]{fontenc}
  \usepackage[utf8]{inputenc}
  \usepackage{textcomp} % provide euro and other symbols
\else % if luatex or xetex
  \usepackage{unicode-math} % this also loads fontspec
  \defaultfontfeatures{Scale=MatchLowercase}
  \defaultfontfeatures[\rmfamily]{Ligatures=TeX,Scale=1}
\fi
\usepackage{lmodern}
\ifPDFTeX\else
  % xetex/luatex font selection
\fi
% Use upquote if available, for straight quotes in verbatim environments
\IfFileExists{upquote.sty}{\usepackage{upquote}}{}
\IfFileExists{microtype.sty}{% use microtype if available
  \usepackage[]{microtype}
  \UseMicrotypeSet[protrusion]{basicmath} % disable protrusion for tt fonts
}{}
\makeatletter
\@ifundefined{KOMAClassName}{% if non-KOMA class
  \IfFileExists{parskip.sty}{%
    \usepackage{parskip}
  }{% else
    \setlength{\parindent}{0pt}
    \setlength{\parskip}{6pt plus 2pt minus 1pt}}
}{% if KOMA class
  \KOMAoptions{parskip=half}}
\makeatother
% Make \paragraph and \subparagraph free-standing
\makeatletter
\ifx\paragraph\undefined\else
  \let\oldparagraph\paragraph
  \renewcommand{\paragraph}{
    \@ifstar
      \xxxParagraphStar
      \xxxParagraphNoStar
  }
  \newcommand{\xxxParagraphStar}[1]{\oldparagraph*{#1}\mbox{}}
  \newcommand{\xxxParagraphNoStar}[1]{\oldparagraph{#1}\mbox{}}
\fi
\ifx\subparagraph\undefined\else
  \let\oldsubparagraph\subparagraph
  \renewcommand{\subparagraph}{
    \@ifstar
      \xxxSubParagraphStar
      \xxxSubParagraphNoStar
  }
  \newcommand{\xxxSubParagraphStar}[1]{\oldsubparagraph*{#1}\mbox{}}
  \newcommand{\xxxSubParagraphNoStar}[1]{\oldsubparagraph{#1}\mbox{}}
\fi
\makeatother


\usepackage{longtable,booktabs,array}
\usepackage{calc} % for calculating minipage widths
% Correct order of tables after \paragraph or \subparagraph
\usepackage{etoolbox}
\makeatletter
\patchcmd\longtable{\par}{\if@noskipsec\mbox{}\fi\par}{}{}
\makeatother
% Allow footnotes in longtable head/foot
\IfFileExists{footnotehyper.sty}{\usepackage{footnotehyper}}{\usepackage{footnote}}
\makesavenoteenv{longtable}
\usepackage{graphicx}
\makeatletter
\newsavebox\pandoc@box
\newcommand*\pandocbounded[1]{% scales image to fit in text height/width
  \sbox\pandoc@box{#1}%
  \Gscale@div\@tempa{\textheight}{\dimexpr\ht\pandoc@box+\dp\pandoc@box\relax}%
  \Gscale@div\@tempb{\linewidth}{\wd\pandoc@box}%
  \ifdim\@tempb\p@<\@tempa\p@\let\@tempa\@tempb\fi% select the smaller of both
  \ifdim\@tempa\p@<\p@\scalebox{\@tempa}{\usebox\pandoc@box}%
  \else\usebox{\pandoc@box}%
  \fi%
}
% Set default figure placement to htbp
\def\fps@figure{htbp}
\makeatother





\setlength{\emergencystretch}{3em} % prevent overfull lines

\providecommand{\tightlist}{%
  \setlength{\itemsep}{0pt}\setlength{\parskip}{0pt}}



 
\usepackage[]{natbib}
\bibliographystyle{agsm}


\addtolength{\oddsidemargin}{-.5in}%
\addtolength{\evensidemargin}{-1in}%
\addtolength{\textwidth}{1in}%
\addtolength{\textheight}{1.7in}%
\addtolength{\topmargin}{-1in}%
\makeatletter
\@ifpackageloaded{caption}{}{\usepackage{caption}}
\AtBeginDocument{%
\ifdefined\contentsname
  \renewcommand*\contentsname{Table of contents}
\else
  \newcommand\contentsname{Table of contents}
\fi
\ifdefined\listfigurename
  \renewcommand*\listfigurename{List of Figures}
\else
  \newcommand\listfigurename{List of Figures}
\fi
\ifdefined\listtablename
  \renewcommand*\listtablename{List of Tables}
\else
  \newcommand\listtablename{List of Tables}
\fi
\ifdefined\figurename
  \renewcommand*\figurename{Figure}
\else
  \newcommand\figurename{Figure}
\fi
\ifdefined\tablename
  \renewcommand*\tablename{Table}
\else
  \newcommand\tablename{Table}
\fi
}
\@ifpackageloaded{float}{}{\usepackage{float}}
\floatstyle{ruled}
\@ifundefined{c@chapter}{\newfloat{codelisting}{h}{lop}}{\newfloat{codelisting}{h}{lop}[chapter]}
\floatname{codelisting}{Listing}
\newcommand*\listoflistings{\listof{codelisting}{List of Listings}}
\makeatother
\makeatletter
\makeatother
\makeatletter
\@ifpackageloaded{caption}{}{\usepackage{caption}}
\@ifpackageloaded{subcaption}{}{\usepackage{subcaption}}
\makeatother
\usepackage{bookmark}
\IfFileExists{xurl.sty}{\usepackage{xurl}}{} % add URL line breaks if available
\urlstyle{same}
\hypersetup{
  pdftitle={Begin With Bayes: Assessing a Probabilistic and Decision Theoretic Approach to Statistics and Machine Learning Instruction},
  pdfauthor={Marc Dotson; Reagan Siggard; Tyler Brough},
  pdfkeywords={Pedagogy, Probabilistic Machine Learning, Statistics
Instruction, Machine Learning Instruction, Bayesian
Inference, Probability, Decision Theory},
  colorlinks=true,
  linkcolor={blue},
  filecolor={Maroon},
  citecolor={Blue},
  urlcolor={Blue},
  pdfcreator={LaTeX via pandoc}}



\begin{document}


\def\spacingset#1{\renewcommand{\baselinestretch}%
{#1}\small\normalsize} \spacingset{1}


%%%%%%%%%%%%%%%%%%%%%%%%%%%%%%%%%%%%%%%%%%%%%%%%%%%%%%%%%%%%%%%%%%%%%%%%%%%%%%

\date{April 9, 2026}
\title{\bf Begin With Bayes: Assessing a Probabilistic and Decision
Theoretic Approach to Statistics and Machine Learning Instruction}
\author{
Marc Dotson\\
Utah State University\\
and\\Reagan Siggard\\
Utah State University\\
and\\Tyler Brough\\
Utah State University\\
}
\maketitle

\bigskip
\bigskip
\begin{abstract}
The framing and depth of statistics and machine learning instruction
varies widely. These differences rightly depend on the specifics of the
field in which statistics and machine learning is applied, aligning
instruction and existing student understanding. However, in settings
where students don't have a common application, statistics and machine
learning instruction and existing student understanding can be
misaligned. In this paper, we propose using probability and decision
theory as a general, unifying approach to statistics and machine
learning instruction. We evaluate the impact of this approach on
business school student understanding in an interdiscplinary setting.
\end{abstract}

\noindent%
{\it Keywords:} Pedagogy, Probabilistic Machine Learning, Statistics
Instruction, Machine Learning Instruction, Bayesian
Inference, Probability, Decision Theory
\vfill

\newpage
\spacingset{1.9} % DON'T change the spacing!


\section{Introduction}\label{sec-intro}

The framing and depth of statistics and machine learning instruction
varies widely. These differences rightly depend on the specifics of the
field in which statistics and machine learning is applied, aligning
instruction and existing student understanding. However, in settings
where students don't have a common application, statistics and machine
learning instruction and existing student understanding can be
misaligned. In this paper, we propose using probability and decision
theory as a general, unifying approach to statistics and machine
learning instruction. We evaluate the impact of this approach on
business school student understanding in an interdiscplinary setting.

\subsection{Decision Theory}\label{decision-theory}

Decision theory is a principled, unifying framework for informing
decision making in the presence of uncertainty. While we can and should
provide a variety of business ``cases'' to illustrate applications, we
can't cover every possible application. Decision theory facilitates the
generalization of a principled approach to any kind of application. It
also helps embed data analysts in the details of the business problem so
they can better communicate the implications of their analysis to the
decision makers. Decision theory is clearly not used in every
discipline, but it can be used in every discipline. This framework can
both help differentiate our students and improve the practice of data
analytics.

To illustrate how decision theory can be used, let's consider a
canonical marketing application of deciding on how to set optimal
prices. In the presence of uncertainty about heterogeneity in consumer
preferences and price sensitivity, businesses actively price
discriminate through targeted promotional offers (e.g., coupons). As a
data analyst communicates with domain experts to understand the problem
and how to inform the decision, a decision theoretic approach requires
specifying an objective function that is consistent with the problem and
captures the risk aversion of the decision maker. For this problem, a
simple customer-specific profit function suffices:

\[
\pi(p) = Y(p) \times (p – mc)
\]

where p is the personalized price for the given customer, Y(p) is the
predicted demand for the given customer, and mc is the marginal cost. In
decision theory, the objective function is typically a loss function. We
can identify an optimal decision, conditioned on the specification of
the objective function, by minimizing the loss function or equivalently,
as we have here, maximizing a utility or profit function.

After working with domain experts to specify an objective function, we
quantify uncertainty about heterogeneity in consumer preferences and
price sensitivity by modeling demand. As with any modeling application,
we end up comparing models. Instead of comparing models using in-sample
fit or even out-of-sample, predictive fit -- with decision theory, we
compare models based on the specified objective function that directly
relates to the decision we're trying to inform. This decision theoretic
approach, which can similarly be used to inform the choice of estimators
in an analysis, can result in the selection of models (and estimators)
that differ widely from those that would be selected using a purely
statistical approach.

For more details on this example, see Smith et al.'s Optimal Price
Targeting paper in Marketing Science.

\subsection{Probability}\label{probability}

A probabilistic approach to machine learning enhances the principled,
unifying framework provided by decision theory. This includes Bayesian
statistics, where we treat all unknowns as random variables and can
directly use probability distributions to quantify uncertainty in our
estimates, and frequentist statistics, where we treat data as random and
indirectly use probability distributions to quantify uncertainty in our
estimates. The direct, Bayesian approach is arguably more intuitive,
especially for applied students, while the indirect, frequentist
approach is less computationally intensive. The hope is that introducing
them in contrast and as complements will enhance student understanding
of modeling generally and combat the problem of student overfitting to a
set list of procedures.

For more details on this approach, see Kevin Murphy's Probabilistic
Machine Learning: An Introduction and Probabilistic Machine Learning:
Advanced Topics.

\section{Study}\label{sec-study}

In this study, we seek to measure the impact of employing an approach to
machine learning instruction tailored to applied and interdisciplinary
students. Specifically, we are interested in exploring whether a
probabilistic and decision theoretic approach to machine learning
improves business student understanding of and confidence in using
machine learning in practice. By evaluating student understanding of
learning objectives this semester as a baseline, we hope to assess the
impact of changes to our approach in subsequent semesters.

\subsection{Steps}\label{steps}

\begin{enumerate}
\def\labelenumi{\arabic{enumi}.}
\tightlist
\item
  The students will receive the invitation to participate in the
  research via the Recruiting Announcement on their Canvas course.
  (Reagan Siggard will post this in the in person and online course.)
\item
  The students will or will not complete the informed consent that is
  housed on Reagan Siggard's Qualtrics Survey account.
\item
  Students who do or do not participate in the study will complete the
  pre-class survey.
\item
  Students who do or do not participate in the study will complete the
  post-class survey.
\item
  The Qualtrics survey will be housed within Reagan's Qualtrics account
  so the PI is unable to view who consented to the survey. 6. Reagan
  will compile the list of students who participated and those who chose
  the alternative extra credit assignment. Reagan will share this list
  with the course instructors at the end of the semester. It will not be
  distinguishable as to which students participated in the research and
  who chose the alternative option. The instructors will then add the
  extra credit.
\item
  After the class has concluded, all data from the pre and post survey
  will be downloaded (identifiable), and those students who opted out of
  the study responses will be removed.
\item
  After the class has concluded, only student outcome data (grades,
  assignments, etc.) of those who participated in the study will be
  downloaded (identifiable) and uploaded to a USU Box Folder.
\item
  We will replace names with a pseudonym and a-numbers will be removed
  from the survey data.
\item
  The research team will store all data in a USU Box folder only shared
  amongst the research team.
\end{enumerate}

\section{Results}\label{sec-results}

Our tables and figures can be dynamically generated.

\section{Conclusion}\label{sec-concl}

One final note. While the bibliography will be placed automatically at
the end of the paper, we may have a few additional citations like R
packages and other software to include that aren't explicitly cited
elsewhere that we can include using the LaTeX syntax
\texttt{\textbackslash{}nocite\{author:year,\ author:year\}} (since we
are using \texttt{natbib} for citations) as is demonstrated in the
following bibliography.

\nocite{bayesm:2018, loo:2018}


\bibliography{references.bib}



\end{document}
